\documentclass{book}
\usepackage[utf8]{inputenc}
\usepackage{enumitem}
\usepackage{amsmath}

\title{Cours de maths synthétique (pour CCINP)}

\begin{document}

\maketitle

\tableofcontents

\chapter{Algèbre}

\section{Propriétés sur les inverses d'une matrice}

\begin{itemize}[label=$\ast$]
    	\item Deux matrices diagonales commutent
    	\item Si A est une matrice triangulaire supérieure et inversible, alors $A^{-1}$ est triangulaire supérieure
	\item Soit A et B deux parties de E. Si A est une partie génératrice de E et si \( A \subset B \) alors B est une parties génératrice de E (c'est la même chose pour les familles)
	\item \( (x,y) \) liée signifie que \( x \) et \( y \) sont colinéaires.
	\item Toutes sous-famille d'une famille finie libre est libre
	\item Toutes sous-famille d'une famille finie liée est liée
	\item Une familel finie de polynômes non nuls de degrés 2 à 2 distincts est libre
	\item Une application \(u \in L(E,F) \) est injective ssi \(Ker u = \{0\} \)
	\item Une symétrie s est un automorphisme donc \(s^{-1} = s \)
	\item Soit H un SEV de E. Alors H est un hyperplan de E ssi il existe une droite vectorielle D telle que \(E = H \oplus D \)
	\item \underline{Théorème de la base incomplète} : Toute famille libre fini d'un EV de dimension fini peut être complété en une base de cet EV
	\item \underline{Théorème de la base extraite} : Toutes familles génératrice finie d'un EV fini on peut extraire une base de cet EV
\end{itemize}

\subsection{}

Ceci est une sous-section, qui est une division plus petite d'une section.

\chapter{Développement}

Ceci est le second chapitre. Vous pouvez y développer votre sujet principal.

\section{Une Autre Section}

Chaque chapitre peut avoir plusieurs sections.

\end{document}

