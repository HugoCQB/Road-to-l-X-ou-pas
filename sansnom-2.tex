\documentclass{article}
\usepackage[utf8]{inputenc}
\usepackage{enumitem}
\usepackage{amsmath}
\usepackage[normalem]{ulem}
\usepackage{eurosym}

\title{Cours de maths synthétique (pour \sout{X/ENS} CCINP/CENTRALE/MINES)}

\author{Crée par des gens de la PSI de Claude B}


\begin{document}

\maketitle

\newpage

\tableofcontents

\newpage

\section{Algèbre}

\subsection{Non assigné pour l'instant}
\begin{itemize}[label=$\ast$]
	\item Soit A et B deux parties de E. Si A est une partie génératrice de E et si \( A \subset B \) alors B est une parties génératrice de E (c'est la même chose pour les familles)
	\item Une symétrie s est un automorphisme donc \(s^{-1} = s \)
	\item 	\( Mat_B(v \circ u) = Mat_B(v)Mat_B(u) \)
	\item \underline{inégalité triangulaire} : \( \|a + b\| \leq \|a\| + \|b\| \)
	\item \underline{théorème de pythagore} : \( \|x + y\| = \|x\| + \|y\| \) ssi x et y sont orthogonaux
	\item Toute famille orthogonale de vecteurs non nuls de E est libre, en particulier, toute famille orthonormée de E est libre
	\item \underline{Théorème de la base orthonormée incomplète} : Toute famille orthonormée de E peut être complétée en une base orthonormée de E
	\item \underline{Théorème de la base incomplète} : Toute famille libre fini d'un EV de dimension fini peut être complété en une base de cet EV
	\item \underline{Théorème de la base extraite} : Toutes familles génératrice finie d'un EV fini on peut extraire une base de cet EV
	\item Toute espace euclidien possède une base orthonormée
	\item \textbf{Formule de Leibniz pour les polynômes :} Pour \( P, Q \in K[X] \) et \( n \in N \), alors \[ (PQ)^{(n)} = \sum_{k=0}^{n} \binom{n}{k} P^{(k)}Q^{(n-k)} \].
	\item \textbf{Formule de Taylor pour les polynômes :} Pour \( P \in K[X] \) et \( a \in K \), alors \[ P(X) = \sum_{n \geq 0} \frac{P^{(n)}(a)}{n!}(X - a)^n \].
	\item \textbf{Théorème (division euclidienne des polynômes) :} Soient \( A, B \in K[X] \) avec \( B \) non nul. Il existe un unique couple \( (Q, R) \in K[X] \) tel que \( A = BQ + R \) et \( \deg(R) < \deg(B) \).



\end{itemize}

\subsection{Caractérisations sur l'inversibilité d'une matrice}
A \( \in M_{n,p}(K) \) (par défaut) est inversible \underline{SI ET SEULEMENT SI} : 
\begin{itemize}[label=$\ast$]
	\item A est de determinant non nul
	\item 0 n'est pas valeur propre de A
	\item Ker(A)= \{0\}
	\item Il existe un polynôme annulateur de A dont 0 n'est pas racine
	\item ses coefficients diagonaux sont tous non nuls dans le cas d'une matrice triangulaire (et son inverse est elle aussi triangulaire supérieure)
	\item \(rg(A) = n\) si \(A \in M_n(K)\)
	\item \( \forall X \in K^n\) \(AX=0 \rightarrow X=0\) si \( A \in M_n(K) \) 
	\item \(\exists B \in M_n(K), AB = I_n\), alors \(A \in M_n(K)\) et B sont inversibles et inverse l'une de l'autre
\end{itemize}

\subsection{Propriété sur la semblablilité de deux matrices}
A et B sont semblables \underline{SI ET SEULEMENT SI} : 
\begin{itemize}[label=$\ast$]
	\item \( det(A) = det(B) \)
	\item \( tr(A) = tr(B) \)
	\item \( rg(A) = rg(B) \)
\end{itemize}

\subsection{Propriété sur la diagonalisation de matrice}
\begin{itemize}[label=$\ast$]
	\item Si le polynôme caractéristique de u est scindé à racines simples, alors u est diagonalisable
	\item A est diagonalisable \underline{si et seulement si} elle est symétrique réelle
\end{itemize}


\subsection{Formules à connaitre}
\begin{itemize}[label=$\ast$]
	\item Pour \( (A,B) \in M_n(K)^2 \) tel que \(AB = BA\), on a pour tout, \(p \in N\) : \((A+B)^p = \sum_{k=0}^{p} \binom{n}{k} A^k B^{p-k} \) et \(A^p - B^p = (A-B) \sum_{k = 0}^{p-1}A^kB^{p-1-k}\)
\end{itemize}

\subsection{Propriété sur les dimensions}
\begin{itemize}[label=$\ast$]
	\item \( dim(E \times F) = dim(E) + dim(F) \)
	\item \( dim(E \oplus F) = dim(E) + dim(F) \)
	\item \( dim(F + G) = dim(F) + dim(G) - dim(F \cap G) \)
	\item \(dim L(E,F) = dim(E) \times dim(F) \)
	\item \underline{Théorème du rang} : dim(E)  = rg(u) + dim(Ker u)
	\item dim(Ker A) + rg(A) = p avec \(A \in M_{n,p}(K)\)
\end{itemize}

\subsection{Propriété sur la liberté d'une famille}
\begin{itemize}[label=$\ast$]
	\item Toutes sous-famille d'une famille finie liée est liée
	\item Une famille finie de polynômes non nuls de degrés 2 à 2 distincts est libre
	\item Toutes sous-famille d'une famille finie libre est libre
	\item  Toutes famille de vecteurs propres associées à des valeurs propres deux à deux distinctes est libre
	\item \( (x,y) \) liée signifie que \( x \) et \( y \) sont colinéaires.
\end{itemize}

\subsection{Propriété sur le rang d'une matrice}
\begin{itemize}[label=$\ast$]
	\item la trace d'un projecteur est égal à son rang
	\item Une matrice et sa transposée ont même rang
	\item Une matrice est de rang r ssi elle est équivalente à la matrice \( J_r \)
	\item Deux matrices de même taille sont équivalentes ssi elles ont même rang
	\item une sous matrice de A a un rang inférieur à A
\end{itemize}

\subsection{Propriété sur la trace d'une matrice}
\begin{itemize}[label=$\ast$]
	\item tr(AB) = tr(BA)
	\item \(tr(v \circ u) = tr(u \circ v) \)
\end{itemize}

\subsection{Propriété sur l'hyperplan}
\begin{itemize}[label=$\ast$]
	\item H hyperplan de E \( \leftrightarrow dim(H) = dim(E)-1 \)
	\item Soit H un SEV de E. Alors H est un hyperplan de E ssi il existe une droite vectorielle D telle que \(E = H \oplus D \)
\end{itemize}

\subsection{Propriété sur l'injectivité, la surjectivité et la bijectivité d'une application}
\begin{itemize}[label=$\ast$]
	\item Une application \(u \in L(E,F) \) est injective ssi \(Ker(u) = \{0\} \)
	\item Soit E et F deux EV de même dimensions finie. On a que \(u \in L(E,F) \) on a :  u surjective \( \Leftrightarrow \) u injective \( \Leftrightarrow \) u bijective
\end{itemize}

\subsection{Propriété sur les valeurs propres, vecteurs propres et les sous espaces propres}
\begin{itemize}[label=$\ast$]
	\item Un scalaire \( \lambda \in K \) est une valeur propre de A ssi il est racine du polynôme caractéristique de A
	\item Si deux endomorphismes commutent, les sous espaces propres de l'un sont stable par l'autre
	\item Si A est une matrice triangulaire, alors l'ensemble de ses valeurs propres est sa diagonale
	\item Deux matrices semblambles ont le même spectre et les sous-espaces propres associés sont de même dimension
	\item on a pour tout \( \lambda \in Sp(u) \) : \( 1 \leq dim(E_{\lambda}(u) \leq m(\lambda) \)
	\item \underline{Théorème de Cayley-Hamilton} : Le polynôme caractéristique de u annule u
	\item Si \( (\lambda_i)_{i \in I} \) est une famille finie de valeurs propres de u deux à deux distinctes, alors les sous espaces propres associées \(E_{\lambda_{i}}(u)\) , pour \(i \in I \) , sont en somme directe.
\end{itemize}

\subsection{Propriété sur les EV de dimension fini}

\section{Analyse}

\subsection{Propriétés non triées encore}

\begin{itemize}[label=$\ast$]
	\item \( \sum_{k=1}^{n}k = \frac{n(n+1)}{2} \)
	\item \underline{Somme des termes d'une suite arithmétique} : Soit \( (u_k)_{k \in N} \) une suite arithmétique. Alors pour \( (p,n) \in N^2 \) tel que \( p \leq n \), : \(  \sum_{k=p}^{n}u_k = (n-p+1) \frac{u_p + u_n}{2} \)
	\item Pour \(n \in N \) : \( \sum_{k=1}^n k^2 = \frac{n(n+1)(2n+1)}{6} \)
	\item \underline{Somme des termes d'une suite géométrique} : Soit \((u_n)_{n \in N} \) une suite géométrique de raison \( a \neq 1\). Alors \( \forall (p,n) \in N^2 \) tel que \(p \leq n \) : \( \sum_{n=p}^n u_k = \frac{u_p - u_{n+1}}{1-a} \)
	\item \underline{Relation de Pascal} : \( \forall{(n,p) \in N* \times Z} : \binom{n}{p} = \binom{n-1}{p} + \binom{n-1}{p-1} \)
	\item \underline{Symétrie du coefficient binomial} : \( \forall{(n,p) \in N \times Z} : \binom{n}{p} = \binom{n}{n-p} \)
	\item \underline{Formule du binôme de Newton} : soit \((a,b) \in R^2\) et \(n \in N\) : \((a+b)^n = \sum_{p=0}n \binom{n}{p}a^p b^{n-p} \)
	\item Toute partie non vide et majorée (resp. minorée) de R possède une borne supérieure (resp. inférieure)
	\item \underline{Caractérisation de la borne supérieure} : Soit A une partie de R et \(a \in R \). Alors on a a = Sup(A) \underline{si, et seulement si} : \( \forall x \in A, x \leq a\) et \( \forall b<a, \exists x \in A, b<x \)
	\item \underline{Théorème des suites adjacentes} : Si deux suites sont adjacentes, alors elles convergent vers une limite commune
	\item \underline{Théorème de Bolzano Weierstrass} : Toute suite bornée possède au moins une sous-suite convergente
 	\item Une partei A de R est dense dans R, \underline{si, et seulement si}, pour tout réel x, on peut trouver une suite d'éléments de A qui convergent vers x
	\item L'ensemble des suites bornées est stable par somme et par produit
	\item Toute suite convergente est bornée
	\item \textbf{Fonction \( k \)-lipschitzienne} : Soit \( f: I \to R \) une fonction et \( k \in R^+ \). Alors \( f \) est \( k \)-lipschitzienne si et seulement si pour tout \( x, y \in I \), on a : \( |f(x) - f(y)| \leq k |x - y| \).
	\item \underline{Formule d'intégration par parties} : \( \forall u, v \in C^1(\mathcal{I}), \quad \int u \, dv = uv - \int v \, du \)
	\item \underline{Formule de Taylor avec reste intégrale} : \( f(x) = \sum_{k=0}^{n} \frac{f^{(k)}(a)}{k!}(x-a)^k + \int_{a}^{x} \frac{f^{(n+1)}(t)}{(n)!}(x-t)^{n} \, dt \)
	\item \underline{inégalité de Taylor Lagrange} : 
	\item \textbf{Limite en un point :} Soit \( f: I \rightarrow R \) une fonction, \( a \) un point de \( I \) ou une extrémité de \( I \), et \( \ell \in R \). On dit que \( f \) admet pour limite \( \ell \) en \( a \) si pour tout \( \epsilon > 0 \), il existe \( \eta > 0 \) tel que pour tout \( x \in I \) avec \( |x - a| < \eta \), on a \( |f(x) - \ell| < \epsilon \).

	\item \textbf{Limite en \( +\infty \) :} Soit \( f: I \rightarrow R \) une fonction, \( a \) une extrémité de \( I \). On dit que \( f \) admet pour limite \( +\infty \) en \( a \) si pour tout \( M > 0 \), il existe \( \eta > 0 \) tel que pour tout \( x \in I \) avec \( |x - a| < \eta \), on a \( f(x) > M \).

	\item \textbf{Limite en \( +\infty \) :} Soit \( f: [a, +\infty[ \rightarrow R \) et \( \ell \in R \). On dit que \( f \) admet pour limite \( \ell \) en \( +\infty \) si pour tout \( \epsilon > 0 \), il existe \( A > 0 \) tel que pour tout \( x \in [a, +\infty[ \) avec \( x \geq A \), on a \( |f(x) - \ell| < \epsilon \).

	\item \textbf{Limite en \( +\infty \) :} Soit \( f: [a, +\infty[ \rightarrow R \). On dit que \( f \) admet pour limite \( +\infty \) en \( +\infty \) si pour tout \( M > 0 \), il existe \( A > 0 \) tel que pour tout \( x \in [a, +\infty[ \) avec \( x \geq A \), on a \( f(x) > M \).

	\item \textbf{Théorème (caractérisation séquentielle de la limite) :} 
Une fonction \( f \) admet pour limite \( \ell \) en \( a \) si et seulement si, pour toute suite \( (x_n) \) qui converge vers \( a \), la suite \( (f(x_n)) \) converge vers \( \ell \).

	\item Si une fonction est dérivable en un point, alors elle est continue en ce point

	\item \textbf{Théorème de Rolle :} Soit \( f : [a, b] \rightarrow R \) une fonction continue sur \([a, b]\), dérivable sur \((a, b)\) et telle que \(f(a) = f(b)\). Alors il existe \(c\) appartenant à \((a, b)\) tel que \(f'(c) = 0\).

	\item \textbf{Théorème des accroissements finis :} Soit \( f : [a, b] \rightarrow R \) une fonction continue sur \([a, b]\) et dérivable sur \((a, b)\). Alors il existe \( c \) appartenant à \((a, b)\) tel que
\[ f(b) - f(a) = f'(c)(b - a) \].

	\item \textbf{Inégalité des accroissements finis :} Soit \( f : [a, b] \rightarrow R \) une fonction continue sur \([a, b]\) et dérivable sur \((a, b)\). Supposons qu'il existe \( M > 0 \) tel que, pour tout \( t \in ]a, b[ \), \( |f'(t)| \leq M \). Alors :
\[ |f(b) - f(a)| \leq M|b - a| \].

	\item \textbf{Formule de Leibniz :} Soient \( f \) et \( g \) deux fonctions \( n \) fois dérivables sur \( I \). Alors \( fg \) est \( n \) fois dérivable sur \( I \) et
\[ (fg)^{(n)} = \sum_{k=0}^{n} \binom{n}{k} f^{(n-k)}g^{(k)} \].

	\item \textbf{Définition de la convexité :} Une fonction \( f \) est dite convexe si, pour tous \( x, y \in I \) et tout \( \lambda \in [0, 1] \), on a
\[ f(\lambda x + (1-\lambda)y) \leq \lambda f(x) + (1-\lambda)f(y) \].

	\item \textbf{Théorème (inégalité des pentes) :} Soit \( f : I \rightarrow R \). Les assertions suivantes sont équivalentes :
1. \( f \) est convexe sur \( I \).
2. Pour tout \( a \in I \), la fonction \( x \mapsto f(x) - f(a)(x - a) \) est croissante sur \( I \setminus \{a\} \).
3. Pour tous \( a, b, c \in I \) avec \( a < b < c \), on a
   \[ f(b) - f(a) \frac{b - a}{b - c} \leq f(c) - f(a) \frac{c - a}{b - c} \leq f(c) - f(b) \frac{c - b}{b - c}. \]

	\item \textbf{Théorème (inégalité de Jensen) :} Une fonction \( f \) est convexe si et seulement si, pour tout \( n \geq 2 \), pour tous \( x_1, \ldots, x_n \in I \) et pour tous les réels \( \lambda_1, \ldots, \lambda_n \) de \( [0, 1] \) tels que \( \sum_{i=1}^{n} \lambda_i = 1 \), alors
\[ f\left(\sum_{i=1}^{n} \lambda_i x_i\right) \leq \sum_{i=1}^{n} \lambda_i f(x_i) \].





\end{itemize}


\section{Probabilité}
	
Soient \( E \) et \( F \) deux ensembles finis. Alors

\begin{itemize}
	\item Si \( E \subseteq F \), on a \( \text{card}(E) \leq \text{card}(F) \), avec égalité si et seulement si \( E = F \).
	\item \( \text{card}(E \times F) = \text{card}(E) \times \text{card}(F) \).
	\item \( \text{card}(E \cup F) = \text{card}(E) + \text{card}(F) - \text{card}(E \cap F) \).
	\item Le cardinal des applications de \( E \) dans \( F \) vaut \( (\text{card} \, F)^{\text{card}(E)} \).
	\item \( \text{card}(\mathcal{P}(E)) = 2^{\text{card}(E)} \).
	\item $P(\emptyset) = 0;$
	\item Pour tout $A \in \mathcal{P}(\Omega)$, $P(\overline{A}) = 1 - P(A);$
	\item Pour tous $A, B \in \mathcal{P}(\Omega)$, $A \subset B \Rightarrow P(A) \leq P(B);$
 	\item Pour tous $A, B \in \mathcal{P}(\Omega)$, $P(A \cup B) = P(A) + P(B) - P(A \cap B);$
	\item Pour toute famille $A_1, \ldots, A_p$ d'événements deux à deux incompatibles, $P(A_1 \cup \cdots \cup A_p) = P(A_1) + \cdots + P(A_p);$
	\item Pour tout système complet d'événements $A_1, \ldots, A_p$, $P(A_1 \cup \cdots \cup A_p) = 1.$
	\item Si \( A_1, \ldots, A_n \) sont des événements mutuellement indépendants, et si pour chaque \( i \in \{1, \ldots, n\} \), on pose \( B_i = A_i \) ou \( B_i = \overline{A_i} \), alors les événements \( B_1, \ldots, B_n \) sont mutuellement indépendants.
	\item \textbf{Proposition :} Si \( B \) est un événement tel que \( P(B) > 0 \), alors \( P_B \) est une probabilité sur \( \Omega \).

	\item \textbf{Formule des probabilités composées :} Soit \( A_1, \ldots, A_m \) des événements tels que \( P(A_1 \cap \cdots \cap A_{m-1}) \neq 0 \). Alors : \[ P(A_1 \cap \cdots \cap A_m) = P(A_1) P(A_2 | A_1) P(A_3 | A_1 \cap A_2) \cdots P(A_m | A_1 \cap \cdots \cap A_{m-1}) \].
	\item \textbf{Formule des probabilités totales :} Soit \( A_1, \ldots, A_n \) un système complet d'événements, tous de probabilité non nulle. Soit \( B \) un événement. Alors :
\[ P(B) = \sum_{i=1}^{n} P(A_i) P(B | A_i) \].
	\item \textbf{Formule de Bayes pour deux événements :} Si \( A \) et \( B \) sont deux événements de probabilité non nulle, alors
\[ P(A | B) = \frac{P(B | A) P(A)}{P(B)} \].
	\item \textbf{Proposition :} Si \( B \) est un événement tel que \( P(B) > 0 \), alors \( P_B \) est une probabilité sur \( \Omega \).
	\item \textbf{Formule des probabilités composées :} Soit \( A_1, \ldots, A_m \) des événements tels que \( P(A_1 \cap \cdots \cap A_{m-1}) \neq 0 \). Alors :
\[ P(A_1 \cap \cdots \cap A_m) = P(A_1) P(A_2 | A_1) P(A_3 | A_1 \cap A_2) \cdots P(A_m | A_1 \cap \cdots \cap A_{m-1}) \].

	\item \textbf{Formule des probabilités totales :} Soit \( A_1, \ldots, A_n \) un système complet d'événements, tous de probabilité non nulle. Soit \( B \) un événement. Alors :
\[ P(B) = \sum_{i=1}^{n} P(A_i) P(B | A_i) \].

	\item Si \( X \) et \( Y \) sont indépendantes, alors \( E(XY) = E(X) E(Y) \).
    \item \( V(X) = E(X^2) - E(X)^2 \);
    \item \( V(aX + b) = a^2 V(X) \).




\textbf{Propriétés des probabilités :}
\begin{enumerate}
    \item \( P(\emptyset) = 0 \);
    \item Pour tout \( A \in \mathcal{P}(\Omega) \), \( P(\overline{A}) = 1 - P(A) \);
    \item Pour tous \( A, B \in \mathcal{P}(\Omega) \), \( A \subseteq B \implies P(A) \leq P(B) \);
    \item Pour tous \( A, B \in \mathcal{P}(\Omega) \), \( P(A \cup B) = P(A) + P(B) - P(A \cap B) \);
    \item Pour toute famille \( A_1, \ldots, A_p \) d'événements deux à deux incompatibles, 
          \( P(A_1 \cup \cdots \cup A_p) = P(A_1) + \cdots + P(A_p) \);
    \item Pour tout système complet d'événements \( A_1, \ldots, A_p \), \( P(A_1 \cup \cdots \cup A_p) = 1 \).
	\item \textbf{Inégalité de Bienaymé-Tchebychev :} Soit \( X \) une variable aléatoire réelle et soit \( \epsilon > 0 \). Alors 
\[ P(|X - E(X)| \geq \epsilon) \leq \frac{V(X)}{\epsilon^2} \].
	\item \textbf{Inégalité de Markov :} Soit \( X \) une variable aléatoire réelle et soit \( t > 0 \). Alors
\[ P(|X| \geq t) \leq \frac{|E(X)|}{t} \].

\end{enumerate}



\end{itemize}




\end{document}
