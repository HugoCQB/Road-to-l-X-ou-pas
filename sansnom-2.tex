\documentclass{article}
\usepackage[utf8]{inputenc}
\usepackage{enumitem}
\usepackage{amsmath}
\usepackage[normalem]{ulem}
\usepackage{eurosym}

\title{Cours de maths synthétique (pour \sout{X/ENS} CCINP/CENTRALE/MINES)}

\author{Crée par des gens de la PSI de Claude B}


\begin{document}

\maketitle

\newpage

\tableofcontents

\newpage

\section{Algèbre}

\subsection{Non assigné pour l'instant}
\begin{itemize}[label=$\ast$]
	\item Soit A et B deux parties de E. Si A est une partie génératrice de E et si \( A \subset B \) alors B est une parties génératrice de E (c'est la même chose pour les familles)
	\item Une symétrie s est un automorphisme donc \(s^{-1} = s \)
	\item 	\( Mat_B(v \circ u) = Mat_B(v)Mat_B(u) \)
	\item \underline{inégalité triangulaire} : \( \|a + b\| \leq \|a\| + \|b\| \)
	\item \underline{théorème de pythagore} : \( \|x + y\| = \|x\| + \|y\| \) ssi x et y sont orthogonaux
	\item Toute famille orthogonale de vecteurs non nuls de E est libre, en particulier, toute famille orthonormée de E est libre
	\item \underline{Théorème de la base orthonormée incomplète} : Toute famille orthonormée de E peut être complétée en une base orthonormée de E
	\item \underline{Théorème de la base incomplète} : Toute famille libre fini d'un EV de dimension fini peut être complété en une base de cet EV
	\item \underline{Théorème de la base extraite} : Toutes familles génératrice finie d'un EV fini on peut extraire une base de cet EV
	\item Toute espace euclidien possède une base orthonormée
\end{itemize}

\subsection{Caractérisations sur l'inversibilité d'une matrice}
A \( \in M_{n,p}(K) \) (par défaut) est inversible \underline{SI ET SEULEMENT SI} : 
\begin{itemize}[label=$\ast$]
	\item A est de determinant non nul
	\item 0 n'est pas valeur propre de A
	\item Ker(A)= \{0\}
	\item Il existe un polynôme annulateur de A dont 0 n'est pas racine
	\item ses coefficients diagonaux sont tous non nuls dans le cas d'une matrice triangulaire (et son inverse est elle aussi triangulaire supérieure)
	\item \(rg(A) = n\) si \(A \in M_n(K)\)
	\item \( \forall X \in K^n\) \(AX=0 \rightarrow X=0\) si \( A \in M_n(K) \) 
	\item \(\exists B \in M_n(K), AB = I_n\), alors \(A \in M_n(K)\) et B sont inversibles et inverse l'une de l'autre
\end{itemize}

\subsection{Propriété sur la semblablilité de deux matrices}
A et B sont semblables \underline{SI ET SEULEMENT SI} : 
\begin{itemize}[label=$\ast$]
	\item \( det(A) = det(B) \)
	\item \( tr(A) = tr(B) \)
	\item \( rg(A) = rg(B) \)
\end{itemize}

\subsection{Propriété sur la diagonalisation de matrice}
\begin{itemize}[label=$\ast$]
	\item Si le polynôme caractéristique de u est scindé à racines simples, alors u est diagonalisable
	\item A est diagonalisable \underline{si et seulement si} elle est symétrique réelle
\end{itemize}


\subsection{Formules à connaitre}
\begin{itemize}[label=$\ast$]
	\item Pour \( (A,B) \in M_n(K)^2 \) tel que \(AB = BA\), on a pour tout, \(p \in N\) : \((A+B)^p = \sum_{k=0}^{p} \binom{n}{k} A^k B^{p-k} \) et \(A^p - B^p = (A-B) \sum_{k = 0}^{p-1}A^kB^{p-1-k}\)
\end{itemize}

\subsection{Propriété sur les dimensions}
\begin{itemize}[label=$\ast$]
	\item \( dim(E \times F) = dim(E) + dim(F) \)
	\item \( dim(E \oplus F) = dim(E) + dim(F) \)
	\item \( dim(F + G) = dim(F) + dim(G) - dim(F \cap G) \)
	\item \(dim L(E,F) = dim(E) \times dim(F) \)
	\item \underline{Théorème du rang} : dim(E)  = rg(u) + dim(Ker u)
	\item dim(Ker A) + rg(A) = p avec \(A \in M_{n,p}(K)\)
\end{itemize}

\subsection{Propriété sur la liberté d'une famille}
\begin{itemize}[label=$\ast$]
	\item Toutes sous-famille d'une famille finie liée est liée
	\item Une famille finie de polynômes non nuls de degrés 2 à 2 distincts est libre
	\item Toutes sous-famille d'une famille finie libre est libre
	\item  Toutes famille de vecteurs propres associées à des valeurs propres deux à deux distinctes est libre
	\item \( (x,y) \) liée signifie que \( x \) et \( y \) sont colinéaires.
\end{itemize}

\subsection{Propriété sur le rang d'une matrice}
\begin{itemize}[label=$\ast$]
	\item la trace d'un projecteur est égal à son rang
	\item Une matrice et sa transposée ont même rang
	\item Une matrice est de rang r ssi elle est équivalente à la matrice \( J_r \)
	\item Deux matrices de même taille sont équivalentes ssi elles ont même rang
	\item une sous matrice de A a un rang inférieur à A
\end{itemize}

\subsection{Propriété sur la trace d'une matrice}
\begin{itemize}[label=$\ast$]
	\item tr(AB) = tr(BA)
	\item \(tr(v \circ u) = tr(u \circ v) \)
\end{itemize}

\subsection{Propriété sur l'hyperplan}
\begin{itemize}[label=$\ast$]
	\item H hyperplan de E \( \leftrightarrow dim(H) = dim(E)-1 \)
	\item Soit H un SEV de E. Alors H est un hyperplan de E ssi il existe une droite vectorielle D telle que \(E = H \oplus D \)
\end{itemize}

\subsection{Propriété sur l'injectivité, la surjectivité et la bijectivité d'une application}
\begin{itemize}[label=$\ast$]
	\item Une application \(u \in L(E,F) \) est injective ssi \(Ker(u) = \{0\} \)
	\item Soit E et F deux EV de même dimensions finie. On a que \(u \in L(E,F) \) on a :  u surjective \( \Leftrightarrow \) u injective \( \Leftrightarrow \) u bijective
\end{itemize}

\subsection{Propriété sur les valeurs propres, vecteurs propres et les sous espaces propres}
\begin{itemize}[label=$\ast$]
	\item Un scalaire \( \lambda \in K \) est une valeur propre de A ssi il est racine du polynôme caractéristique de A
	\item Si deux endomorphismes commutent, les sous espaces propres de l'un sont stable par l'autre
	\item Si A est une matrice triangulaire, alors l'ensemble de ses valeurs propres est sa diagonale
	\item Deux matrices semblambles ont le même spectre et les sous-espaces propres associés sont de même dimension
	\item on a pour tout \( \lambda \in Sp(u) \) : \( 1 \leq dim(E_{\lambda}(u) \leq m(\lambda) \)
	\item \underline{Théorème de Cayley-Hamilton} : Le polynôme caractéristique de u annule u
	\item Si \( (\lambda_i)_{i \in I} \) est une famille finie de valeurs propres de u deux à deux distinctes, alors les sous espaces propres associées \(E_{\lambda_{i}}(u)\) , pour \(i \in I \) , sont en somme directe.
\end{itemize}

\subsection{Propriété sur les EV de dimension fini}

\section{Analyse}



\section{Probabilité}



\end{document}
